\documentclass{beamer}

\usetheme{Berlin}
\usecolortheme{beaver}
\usepackage[backend=biber, style=numeric]{biblatex}
\usepackage{tabularx}
\newcolumntype{Y}{>{\raggedright\arraybackslash}X}


\addbibresource{biblio.bib}
\title{Digital Twins of the Heart}
\subtitle{Coursework "ROM \& Data-driven ROM" - CSMI}
\author{Rodolphe Vivant \& Antoine Ruch}
\institute{UFR of Mathematics and Informatics}
\date{\today}

\begin{document}

\begin{frame}
  \titlepage
\end{frame}

\begin{frame}{Plan}
  \tableofcontents
\end{frame}

\section{Introduction}

\section{Mathematical Methods}

\subsection{DL/ML Methods}

\begin{frame}{POD-DL-ROM \cite{Fresca2021}}
  \begin{itemize}
    \item FOM - Cardiac Electrophysiology: coupled non-linear dynamical system.
    \item \textbf{Inputs} - ionic model, conductivity... Multiple querries to build multi-scenario analysis.
    \item Proposes a POD-DL-ROM pipeline - collection of FOM snapshots for efficient ROM parameter reconstruction - enhanced with POD.
    \item \textbf{Output} - ACs, APs, ECGs... Fast and efficient evaluation.
  \end{itemize}

  \begin{figure}
      \centering
      \includegraphics[width=1\linewidth]{images/POD-DL-ROM-perf.png}
      \caption{\textcite{Fresca2021}}
  \end{figure}
  
\end{frame}


\begin{frame}{Architecture}
    \begin{figure}
      \centering
      \includegraphics[width=1\linewidth]{images/POD-DL-ROM-archi.png}
      \caption{\textcite{Fresca2021}}
  \end{figure}
\end{frame}

\begin{frame}{PINNs \cite{jafari2022}}
  \begin{itemize}
    \item AI models can extract complex actionable data from time series, but require significant amounts of ground truth data. 
    \item \textbf{PINNs}: neural networks are trained to solve problems leveraging underlying physics of input data.
    \item Builds Taylor's approximations modelling the relationship between physiological features extracted from sensor measurements.
    \item Yields a Taylor approximation in the form of a PDE.
    \item The output of the network is the estimated continuous blood pressure.
  \end{itemize}
\end{frame}

\begin{frame}{Architecture}
    \begin{figure}
      \centering
      \includegraphics[width=1\linewidth]{images/PINNs-archi.png}
      \caption{\textcite{jafari2022}}
  \end{figure}
\end{frame}

\section{Clinical Data \& Continuous Monitoring}
\begin{frame}{Continuous monitoring devices}

  \scriptsize
  \begin{tabularx}{\textwidth}{|Y|c|Y|c|c|Y|c|}
  \hline
  \textbf{Device} & \textbf{Channels} & \textbf{Data} & \textbf{SPS} & \textbf{Bytes} & \textbf{Daily debit / channel} & \textbf{Paper} \\
  \hline
  Max30102 (HR/SpO2) 
  & 2 
  & HR, SpO2 
  & 512 
  & 2 
  & $512 \times 86{,}400 \times 2 \approx 88$ Mb 
  & \cite{jameil2025digital} \\
  \hline
  MLX90614 (IR thermometer) 
  & 1 
  & Body temperature 
  & 28 
  & 2 
  & $28 \times 86{,}400 \times 2 \approx 4.8$ Mb 
  & \cite{jameil2025digital} \\
  \hline
  Finapres NOVA (bio-impedance) 
  & 4 
  & BioZ1--4 
  & 112 
  & 3 
  & $112 \times 86{,}400 \times 3 \approx 260$ Mb 
  & \cite{jafari2022} \\
  \hline
  \end{tabularx}
  
\end{frame}

\section{Verification \& Validation}
\begin{frame}{Strategies \& Metrics}

    \begin{itemize}
        \item Unit tests for solvers and proposed benchmarks - still an emerging field. \textcite{Reidmen2025}
        \item Common good practices for PDE/FEM solvers: convergence studies, numerical stability (e.g. in \textcite{gerach2021electro})
        \item Compare ROM against high-fidelity snapshots - $L^2/L^\infty$ error norm fields. (e.g. in \textcite{Fresca2021})
        \item \textit{RMSE}, \textit{Bland-Altman} and \textit{Pearson’s} correlation analyses for estimated wave forms. (e.g. in \textcite{jafari2022})
        \item Classifier evaluation metrics - Accuracy, Precision, Recall, F-score. (e.g. in \textcite{burak2023})
        \item Clinical trials comparing twin-informed decisions and standard care... (WIP - \href{https://www.imperial.ac.uk/news/253154/digital-twin-heart-modelling-project-will/?utm_source}{CVD-Net})
    \end{itemize}
\end{frame}

\begin{frame}{MRSE \& Correlation analyses}
  
  \begin{figure}
      \centering
      \includegraphics[width=1\linewidth]{images/MRSE.png}
      \caption{\textcite{jafari2022}}
  \end{figure}

\end{frame}

\begin{frame}{Relative error fields}
  
  \begin{figure}
      \centering
      \includegraphics[width=0.7\linewidth]{images/err-fields.png}
      \caption{\textcite{Fresca2021}}
  \end{figure}

\end{frame}

\section{Deployment \& Implementation}

\begin{frame}{Deployment \& Implementation}
  \begin{itemize}
    \item IoT based DT frameworks - continuous measurements from wearable sensors and transmission through the network: bandwidths and latency.
    \item \textit{Offline}: cloud computing (e.g. MS Azure, AWS, GCC...)
    \item \textit{Online}: cloud or edge-based. Low latency crucial for monitoring.
    \item Case study: \textbf{Edge} VS \textbf{Cloud} based DT frameworks for ML classifier modules. Overall better performances of the edge framework (\textcite{burak2023}).
    \item Edge devices: cell-phones, hospital network, embedded computing (e.g. Raspberry Pi).
    \item Additional software development challenges - API for live-monitoring, CI/CD, DevOps/MLOps for maintainability and control loops based on inference metrics.
  \end{itemize}
\end{frame}

\section{Perspectives \& Limits}

\begin{frame}{Perspectives \& Limits}
  
  \textbf{Technical barriers} 
  \begin{itemize}
    \item Computational cost of current methods limits potential reach outside the private health sector. 
    \item Complexity of integrating clinical and sensor data into unified prediction models.
  \end{itemize}
  \textbf{Privacy and ethics}
  \begin{itemize}
    \item Handling of large amounts of sensitive health data. Concerns about complying to health regulations - GDPR, HIPAA... Access to open-source health data for training.
    \item Assessing the responsibility of prediction modules, control and veto mechanisms.
    \item Infrastructure security - edge based frameworks are particularly sensitive (MITM, Ransomware).
  \end{itemize}
\end{frame}

\section*{Déclaration d'assistance IA}
\begin{frame}{Déclaration d'assistance IA}
  Certaines parties de ce travail ont été préparées avec l'assistance d'outils d'Intelligence Artificielle (IA). Les outils suivants ont été utilisés (toutes les sorties vérifiées, révisées et intégrées de manière responsable par les auteurs) :
  \begin{itemize}
    \item ChatGPT (OpenAI, familles GPT-4/5) – assemblage d'une bibliographie introductive comme point de départ pour nos recherches.
  \end{itemize}
  Toutes les sorties IA ont été \textbf{vérifiées, adaptées et corrigées par les auteurs}. Les auteurs acceptent la pleine responsabilité de la justesse, de l'originalité et de l'intégrité scientifique de ce travail.
\end{frame}

\begin{frame}[allowframebreaks]{References}
    \printbibliography
\end{frame}



\end{document}
